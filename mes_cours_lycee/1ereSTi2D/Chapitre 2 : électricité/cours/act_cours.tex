%%%%%%%%%%%%%%%%%%%%%%%%%%%%%%%%%%%%%%%%%
% Cleese Assignment (For Students)
% LaTeX Template
% Version 2.0 (27/5/2018)
%
% This template originates from:
% http://www.LaTeXTemplates.com
%
% Author:
% Vel (vel@LaTeXTemplates.com)
%
% License:
% CC BY-NC-SA 3.0 (http://creativecommons.org/licenses/by-nc-sa/3.0/)
% 
%%%%%%%%%%%%%%%%%%%%%%%%%%%%%%%%%%%%%%%%%

%----------------------------------------------------------------------------------------
%	PACKAGES AND OTHER DOCUMENT CONFIGURATIONS
%----------------------------------------------------------------------------------------



\documentclass[24pt]{article}

\input{activite.sty} % Include the file specifying the document structure and custom commands
\input{annotate_equations.sty}


\newcommand{\prgrmparun}{
    \mypage{
        \begin{itemize}
            \item Citer les différentes formes d’énergie utilisées dans les
                  domaines de la vie courante, de la production et des
                  services.
            \item Distinguer les formes d’énergie des différentes sources
                  d’énergie associées.
        \end{itemize}
    }
}


\newcommand{\prgrmpardeux}{
    \mypage{
        \begin{itemize}
            \item Énoncer et exploiter la relation entre puissance, énergie
                  et durée.
            \item Évaluer et citer des ordres de grandeur des puissances
                  mises en jeu dans les secteurs de l’énergie, de l’habitat,
                  des transports, des communications, etc.
        \end{itemize}
    }
}

\newcommand{\prgrmpartrois}{
    \mypage{
        \begin{itemize}
            \item Identifier les principales conversions d’énergie :
                  électromécanique, photoélectrique, électrochimique,
                  thermodynamique (conversions réalisées par une
                  machine thermique), etc.
            \item Schématiser une chaîne énergétique ou une conversion
                  d’énergie en distinguant formes d’énergie, sources
                  d’énergie et convertisseurs.
            \item Évaluer ou mesurer une quantité d’énergie transférée,
                  convertie ou stockée.
        \end{itemize}
    }
}

\newcommand{\prgrmparquatre}{
    \mypage{
        \begin{itemize}
            \item Énoncer le principe de conservation de l’énergie pour un
                  système isolé.
            \item Exploiter le principe de conservation de l’énergie pour
                  réaliser un bilan énergétique et calculer un rendement
                  pour une chaîne énergétique ou un convertisseur.
            \item Déterminer le rendement d’une chaîne énergétique ou
                  d’un convertisseur
        \end{itemize}
    }
}


\newcommand{\prgrmparcinq}{
    \mypage{
        \begin{itemize}
            \item Énoncer qu’une ressource d’énergie est qualifiée de
                  « renouvelable » si son renouvellement naturel est
                  assez rapide à l’échelle de temps d’une vie humaine.
        \end{itemize}
    }
}


\newcommand{\mypage}[1]{
    \begin{minipage}[t]{0.6\textwidth}
        {#1}
    \end{minipage}
}

%	\setlength{\arrayrulewidth}{0.5mm}
%	\setlength{\tabcolsep}{18pt}
\newcommand{\programme}{
    \renewcommand{\arraystretch}{2}
    \begin{center}
        \begin{tabular}{@{}|l|l|@{}}
            \multicolumn{2}{c}{Programme : Énergie}                     \\ \midrule

            \begin{minipage}[t]{0.3\textwidth}
                {Savoirs}\end{minipage}         & \mypage{Savoirs-faire}    \\\midrule

            Formes d’énergie.                          & \prgrmparun    \\

            Énergie et puissance.                      & \prgrmpardeux  \\

            \begin{minipage}[t]{0.3\textwidth}
                Les conversions et les chaînes
                énergétiques.

                Stockage de l’énergie.\end{minipage}       & \prgrmpartrois \\

            \begin{minipage}[t]{0.3\textwidth}
                Principe de la conservation de l’énergie.
                Rendement\end{minipage}  & \prgrmparquatre                  \\

            Ressource d’énergie dite « renouvelable ». & \prgrmparcinq  \\

            \bottomrule\end{tabular}
    \end{center}

}



%----------------------------------------------------------------------------------------
%	VARIABLES
%----------------------------------------------------------------------------------------

% % Required
% \newcommand{\assignmentQuestionName}{Question} % The word to be used as a prefix to question numbers; example alternatives: Problem, Exercise
\newcommand{\assignmentClass}{Physique Chimie ---} % Course/class
\newcommand{\assignmentTitle}{Chapitre 1: Les éléments chimiques} % Assignment title or name
% \newcommand{\assignmentAuthorName}{Mme Cercy} 
% % \newcommand{\assignmentAuthorName}{Chapitre Y} 
% % \newcommand{\titre}{Activité 1 : Les éléments chimiques dans les étoiles} % titre de l'activité

%----------------------------------------------------------------------------------------

\begin{document}
% \thispagestyle{fancy}
\cfoot{}
\date{}
\title{\huge Chapitre 2 : Énergie électrique}
\maketitle % Print the title page

\programme

\assignmentSection{Cours}

\vspace{-10pt}

\begin{concept}{Loi des nœuds}
    \begin{minipage}[c]{0.75\textwidth}
        La quantité d’électrons qui circulent dans le circuit se conserve. La loi des nœuds traduit
        cette conservation : en C et en G, le courant se divise en deux parties, qui peuvent être
        égales ou non.

        Loi des nœuds : la somme des courants entrant à un nœud est égale à la somme des courants sortant :
        $I_1+I_2=I_3+I_4$

        S’il n’y a pas de nœuds, comme pour deux dipôles associés en série, alors l’intensité reste la même.
    \end{minipage}
    % \hspace{0.1\textwidth}
    \begin{minipage}[c]{0.3\textwidth}
        \begin{center}
            \includegraphics[width=0.4\columnwidth]{noeud.png}
        \end{center}
    \end{minipage}
\end{concept}


\begin{concept}{La tension électrique}
    La tension électrique est une grandeur caractérisant une différence d’état électrique entre
    deux points d’un circuit.

    On a choisi de la représenter par une flèche. Ainsi, dans le circuit modèle,
    la tension $U_2$ est égale à la tension $U_{CF}$ (la flèche pointe vers C).
    La tension $U_{AB}$ (la flèche pointe vers A) est égale à $-U_{BA}$
    (la flèche de $U_{BA}$ pointe vers B).

    La tension U s’exprime en volt noté V.

    La tension électrique aux bornes d’un dipôle se mesure avec un voltmètre
    toujours placé en dérivation sur les bornes de ce dipôle.
\end{concept}

\begin{concept}{circuit électrique}\begin{minipage}[c]{0.75\textwidth}

        Un circuit électrique est composé d’au moins un générateur, un récepteur
        (résistance, moteur, DEL, etc.) et des fils de connexion.

        \begin{itemize}
            \item Un dipôle est un élément d’un circuit électrique possédant deux bornes.
            \item
            \item Un nœud est une connexion qui relie au moins trois dipôles entre eux. Sur
                  le circuit modèle, C et F sont des nœuds électriques.

            \item Une maille est un chemin fermé, ne comportant pas forcément de générateur.
                  Le circuit modèle possède trois mailles : (ABCFA), (ABCDEFA) et (CDEFC).
        \end{itemize}
        \vspace{5pt}

        Il existe deux types d’association des dipôles entre eux,
        l’association en série et l’association en dérivation :
        \begin{list}{$\bullet$}{}
            \item deux dipôles sont en série s'ils sont situés dans la même maille et ne sont pas
                  séparés par un nœud.

            \item deux dipôles sont en dérivation si leurs bornes sont connectées aux mêmes nœuds.
        \end{list}

        Sur le circuit ci dessous, l’ampèremètre et la résistance $R_1$ sont associés en série
        (ils sont donc traversés par un courant de même intensité).
        Les résistances $R_2$ et $R_3$ sont associées en dérivation.
    \end{minipage}
    % \hspace{0.1\textwidth}
    \begin{minipage}[c]{0.25\textwidth}
        \begin{center}
            \includegraphics[width=\columnwidth]{c1.png}
        \end{center}
    \end{minipage}




\end{concept}


\begin{concept}{La loi d'Ohm}
    La loi d’Ohm relie la tension aux bornes d’un résistor (une « résistance ») et
    l’intensité du courant qui le traverse.
    Son expression est: $U = RI$

    \begin{center}
        \includegraphics[width=0.2\columnwidth]{ohm.png}
    \end{center}

    U est exprimée en volt (V), I en ampère (A) et R en ohm ($Ω$). On aura ici $U_1 = R_1 \times I_1$.

    \begin{list}{$\bullet$}{}
        \item Une convention d’écriture importante : pour que les tensions représentées
              correspondent à des valeurs positives de tension, l’orientation des flèches de
              tension est importante. Dans le cas d’un générateur, la flèche représentant la tension
              est orientée dans le même sens que le sens de parcours du courant électrique.

        \item Dans le cas d’un dipôle récepteur passif comme une résistance par exemple, la flèche
              représentant la tension est orientée dans le sens opposé au sens de parcours du courant électrique.
    \end{list}
\end{concept}

\begin{concept}{Loi des mailles}
    \begin{minipage}[c]{0.7\textwidth}
        la somme des tensions des dipôles le long d’une maille est égale à 0 V.

        Ainsi, en parcourant la maille (AGFEDCBA) dans le sens des pointillés verts,
        on peut écrire $U_{AA}=0 V$ soit :
        $$U_{AG}+U_{GF}+U_{FE}+U_{ED}+U_{DC}+U_{CB}+U_{BA}=0 V$$

        $$0+0-U_{EF}-U_{DE}+0+0+U_{BA}=0 V$$

        Par ailleurs, en parcourant la maille (ABCGA) dans le sens des pointillés bleus,
        on peut écrire $U_{AA}=0 V$ soit :
        $$U_{AB}+U_{BC}+U_{CG}+U_{GA}=0V$$
        $$U_{AB}+0+U_{CG}+0=0 V$$

        d’où : $U_{BA}$=$U_{CG}$

        On retrouve ici la loi d’unicité des tensions sur deux branches en dérivation.
    \end{minipage}
    % \hspace{0.1\textwidth}
    \begin{minipage}[c]{0.3\textwidth}
        \begin{center}
            \includegraphics[width=0.8\columnwidth]{c2.png}
        \end{center}
    \end{minipage}
\end{concept}



\end{document}