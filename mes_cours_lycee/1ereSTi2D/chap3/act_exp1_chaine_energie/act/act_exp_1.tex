%%%%%%%%%%%%%%%%%%%%%%%%%%%%%%%%%%%%%%%%%
% Cleese Assignment (For Students)
% LaTeX Template
% Version 2.0 (27/5/2018)
%
% This template originates from:
% http://www.LaTeXTemplates.com
%
% Author:
% Vel (vel@LaTeXTemplates.com)
%
% License:
% CC BY-NC-SA 3.0 (http://creativecommons.org/licenses/by-nc-sa/3.0/)
% 
%%%%%%%%%%%%%%%%%%%%%%%%%%%%%%%%%%%%%%%%%

%----------------------------------------------------------------------------------------
%	PACKAGES AND OTHER DOCUMENT CONFIGURATIONS
%----------------------------------------------------------------------------------------

\documentclass[10pt]{article}
\input{activite.sty} % Include the file specifying the document structure and custom commands
% \usepackage{activite}
%----------------------------------------------------------------------------------------
%	ASSIGNMENT INFORMATION
%----------------------------------------------------------------------------------------

% Required
\newcommand{\assignmentQuestionName}{Question} % The word to be used as a prefix to question numbers; example alternatives: Problem, Exercise
\newcommand{\assignmentClass}{Physique Chimie} % Course/class
\newcommand{\assignmentTitle}{Activité n°} % Assignment title or name
\newcommand{\assignmentAuthorName}{Mme Cercy} %

%----------------------------------------------------------------------------------------
%	VARIABLES
%----------------------------------------------------------------------------------------

\newcommand{\titreActivite}{\huge Activité exp1: Chaine d'énergie} % titre de l'activité

\begin{document}

\date{}
\title{\titreActivite}
\maketitle % Print the title page

% \vspace{-10pt}

\begin{minipage}[c]{0.45\textwidth}
	\centering
	\includegraphics[scale=0.5]{assets/doc1.png}
\end{minipage}
\hspace{20pt}
\begin{minipage}[c]{0.45\textwidth}
	\centering
	\includegraphics[scale=0.36]{assets/doc2.png}
	\includegraphics[scale=0.36]{assets/doc3.png}
\end{minipage}

\includegraphics[scale=0.36]{assets/q.png}

% \begin{documentpeda}{Caractéristiques et protection d’une diode}
% 	La propriété essentielle
% 	d’une diode ...
% \end{documentpeda}

% \begin{question}
% 	\questiontext{
% 		Répondre aux questions suivantes pour...
% 	}
% \end{question}
% \subquestion{...}

\end{document}
