%%%%%%%%%%%%%%%%%%%%%%%%%%%%%%%%%%%%%%%%%
% Cleese Assignment (For Students)
% LaTeX Template
% Version 2.0 (27/5/2018)
%
% This template originates from:
% http://www.LaTeXTemplates.com
%
% Author:
% Vel (vel@LaTeXTemplates.com)
%
% License:
% CC BY-NC-SA 3.0 (http://creativecommons.org/licenses/by-nc-sa/3.0/)
% 
%%%%%%%%%%%%%%%%%%%%%%%%%%%%%%%%%%%%%%%%%

%----------------------------------------------------------------------------------------
%	PACKAGES AND OTHER DOCUMENT CONFIGURATIONS
%----------------------------------------------------------------------------------------

\documentclass[10pt]{article}
\input{activite.sty} % Include the file specifying the document structure and custom commands
% \usepackage{activite}
%----------------------------------------------------------------------------------------
%	ASSIGNMENT INFORMATION
%----------------------------------------------------------------------------------------

% Required
\newcommand{\assignmentQuestionName}{Question} % The word to be used as a prefix to question numbers; example alternatives: Problem, Exercise
\newcommand{\assignmentClass}{Physique Chimie} % Course/class
\newcommand{\assignmentTitle}{Activité n°} % Assignment title or name
\newcommand{\assignmentAuthorName}{Mme Cercy} %

%----------------------------------------------------------------------------------------
%	VARIABLES
%----------------------------------------------------------------------------------------

\newcommand{\titreActivite}{Activité exp 2: rendement d'un moteur} % titre de l'activité

\begin{document}

\date{}
\title{\titreActivite}
\maketitle % Print the title page

\begin{center}
	\includegraphics[scale=0.38]{1.png}
\end{center}
\vspace{-10pt}

\begin{center}
	\includegraphics[scale=0.38]{2.png}
\end{center}

% \begin{center}
% 	\includegraphics[scale=0.38]{q_inspi.png}
% \end{center}

\begin{enumerate}
	\item Sachant que la puissance délivrée au moteur est $P = U \times I$, quelle est l'énergie apportée au moteur?
	      \textbf{Exprimer le résultat} avec t(sec) et P(watt).
	\item La relation pour calculer le rendement s'écrit : $\eta = \displaystyle\frac{\text{Énergie utile}}{\text{Énergie fournie}}$, Quelle est l'énergie utile ? Quelle est l'énergie fournie ?
	\item Calculer le rendement du moteur $\eta$ pour chaque masse puis recopier et compléter le tableau.
	\item Tracer la courbe $\eta = f(m)$
	\item Conclure, le rendement dépend il de la charge?
\end{enumerate}

\end{document}
