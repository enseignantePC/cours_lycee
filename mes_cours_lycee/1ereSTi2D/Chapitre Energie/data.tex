

\newcommand{\prgrmparun}{
    \mypage{
        \begin{itemize}
            \item Citer les différentes formes d’énergie utilisées dans les
                  domaines de la vie courante, de la production et des
                  services.
            \item Distinguer les formes d’énergie des différentes sources
                  d’énergie associées.
        \end{itemize}
    }
}


\newcommand{\prgrmpardeux}{
    \mypage{
        \begin{itemize}
            \item Énoncer et exploiter la relation entre puissance, énergie
                  et durée.
            \item Évaluer et citer des ordres de grandeur des puissances
                  mises en jeu dans les secteurs de l’énergie, de l’habitat,
                  des transports, des communications, etc.
        \end{itemize}
    }
}

\newcommand{\prgrmpartrois}{
    \mypage{
        \begin{itemize}
            \item Identifier les principales conversions d’énergie :
                  électromécanique, photoélectrique, électrochimique,
                  thermodynamique (conversions réalisées par une
                  machine thermique), etc.
            \item Schématiser une chaîne énergétique ou une conversion
                  d’énergie en distinguant formes d’énergie, sources
                  d’énergie et convertisseurs.
            \item Évaluer ou mesurer une quantité d’énergie transférée,
                  convertie ou stockée.
        \end{itemize}
    }
}

\newcommand{\prgrmparquatre}{
    \mypage{
        \begin{itemize}
            \item Énoncer le principe de conservation de l’énergie pour un
                  système isolé.
            \item Exploiter le principe de conservation de l’énergie pour
                  réaliser un bilan énergétique et calculer un rendement
                  pour une chaîne énergétique ou un convertisseur.
            \item Déterminer le rendement d’une chaîne énergétique ou
                  d’un convertisseur
        \end{itemize}
    }
}


\newcommand{\prgrmparcinq}{
    \mypage{
        \begin{itemize}
            \item Énoncer qu’une ressource d’énergie est qualifiée de
                  « renouvelable » si son renouvellement naturel est
                  assez rapide à l’échelle de temps d’une vie humaine.
        \end{itemize}
    }
}


\newcommand{\mypage}[1]{
    \begin{minipage}[t]{0.6\textwidth}
        {#1}
    \end{minipage}
}

%	\setlength{\arrayrulewidth}{0.5mm}
%	\setlength{\tabcolsep}{18pt}
\newcommand{\programme}{
    \renewcommand{\arraystretch}{2}
    \begin{center}
        \begin{tabular}{@{}|l|l|@{}}
            \multicolumn{2}{c}{Programme : Énergie}                     \\ \midrule

            \begin{minipage}[t]{0.3\textwidth}
                {Savoirs}\end{minipage}         & \mypage{Savoirs-faire}    \\\midrule

            Formes d’énergie.                          & \prgrmparun    \\

            Énergie et puissance.                      & \prgrmpardeux  \\

            \begin{minipage}[t]{0.3\textwidth}
                Les conversions et les chaînes
                énergétiques.

                Stockage de l’énergie.\end{minipage}       & \prgrmpartrois \\

            \begin{minipage}[t]{0.3\textwidth}
                Principe de la conservation de l’énergie.
                Rendement\end{minipage}  & \prgrmparquatre                  \\

            Ressource d’énergie dite « renouvelable ». & \prgrmparcinq  \\

            \bottomrule\end{tabular}
    \end{center}

}

