%%%%%%%%%%%%%%%%%%%%%%%%%%%%%%%%%%%%%%%%%
% Cleese Assignment (For Students)
% LaTeX Template
% Version 2.0 (27/5/2018)
%
% This template originates from:
% http://www.LaTeXTemplates.com
%
% Author:
% Vel (vel@LaTeXTemplates.com)
%
% License:
% CC BY-NC-SA 3.0 (http://creativecommons.org/licenses/by-nc-sa/3.0/)
% 
%%%%%%%%%%%%%%%%%%%%%%%%%%%%%%%%%%%%%%%%%

%----------------------------------------------------------------------------------------
%	PACKAGES AND OTHER DOCUMENT CONFIGURATIONS
%----------------------------------------------------------------------------------------

\documentclass[10pt]{article}
\input{activite.sty} % Include the file specifying the document structure and custom commands
% \usepackage{activite}
%----------------------------------------------------------------------------------------
%	ASSIGNMENT INFORMATION
%----------------------------------------------------------------------------------------

% Required
\newcommand{\assignmentQuestionName}{Question} % The word to be used as a prefix to question numbers; example alternatives: Problem, Exercise
\newcommand{\assignmentClass}{Physique Chimie} % Course/class
\newcommand{\assignmentTitle}{Activité n°1:} % Assignment title or name
\newcommand{\assignmentAuthorName}{Mme Cercy} %

%----------------------------------------------------------------------------------------
%	VARIABLES
%----------------------------------------------------------------------------------------

\newcommand{\titreActivite}{\huge TP 1: Loi des noeuds, loi des mailles, loi d'Ohm.} % titre de l'activité

\begin{document}

\date{}
\title{\titreActivite}
\maketitle % Print the title page

% \vspace{-10pt}

\begin{minipage}[c]{0.45\textwidth}
	\begin{documentpeda}{Matériel par binôme}
		\begin{list}{$\bullet$}{}
			% \item Pile de 4,5 V ou de 9 V (6LR61)
			\item Générateur de tension : 12 V
			\item 1 multimètre
			\item Fils de connexion
			\item Interrupteur
			\item Résistances : 2 × 220 $Ω$ et 2 × 470 $Ω$
			\item 2 × DEL
		\end{list}
	\end{documentpeda}
\end{minipage}
\hspace{0.05\textwidth}
\begin{minipage}[c]{0.45\textwidth}
	\begin{center}
		circuit 1 : \vspace{5pt}
		\includegraphics[width=\columnwidth]{c1.png}
	\end{center}
\end{minipage}

% \vspace{-10pt}

\begin{documentpeda}{Caractéristiques et protection d’une diode}
	\begin{minipage}[c]{0.75\textwidth}
		La propriété essentielle
		d’une diode est de ne laisser passer le courant électrique
		que dans un seul sens (le sens passant). Lorsqu’elle émet un
		rayonnement dans le visible, on l’appelle diode électroluminescente ou DEL.

		La diode est un dispositif fragile. L’intensité du courant qui la traverse ne doit pas dépasser 20 mA.
		Pour cette raison, on ajoute un résistor de protection en série avec la diode.
	\end{minipage}
	% \hspace{0.05\textwidth}
	\begin{minipage}[c]{0.26\textwidth}
		\begin{center}
			\includegraphics[width=\columnwidth]{carac.png}
		\end{center}
	\end{minipage}
\end{documentpeda}

\vspace{-10pt}

\begin{question}
	\questiontext{
		Répondre aux questions suivantes pour le schéma du circuit 1:

	}
\end{question}
\subquestion{Reproduire le schéma du montage et flécher les tensions $U_1$, $U_2$, $U_3$ et $U_4$ avec les bornes des résistances.}
\subquestion{Déterminer le nombre de mailles du circuit, les nommer et donner le nom des noeuds électriques.}
\subquestion{Réaliser le circuit en respectant les valeurs de résistances indiquées. Régler la tension du générateur sur 12 V.}
\subquestion{Mesurer toutes les grandeurs tensions et intensités du montage à l'aide de l'appareil de mesure adapté. \textbf{Rassembler} les mesures dans un tableau.}
\subquestion{Expliquer comment procéder pour vérifier que la loi des mailles et la loi des nœuds sont respéctés dans le circuit.}
\subquestion{Procéder à la vérification.}
\subquestion{En utilisant la loi d'Ohm, calculer les valeurs des résistances du circuit et les comparer avec les valeurs marquées sur celles-ci.}
\subquestion{
	Comment expliquer la diminution de luminosité de deux lampes
	dans un montage en série par rapport à un montage en dérivation ?
	Cette explication est-elle toujours valable dans le cas d'un générateur limité en intensité ?
}
\vspace{-20pt}


\begin{question}
	\questiontext{
		Répondre aux questions suivantes pour le schéma du circuit 2:
	} \vspace{-5pt}
\end{question}

%----------------------------------------------------------------------------------------
%	QUESTION 
%----------------------------------------------------------------------------------------
% \vspace{-20pt}



\subquestion{À l’aide de la caractéristique de la diode, indiquer la tension aux bornes de la DEL lorsqu’elle est traversée par un courant maximal de 20 mA.}
\subquestion{À l’aide de la loi des mailles et de la loi d’Ohm, calculer la valeur minimale de la résistance de protection à placer en série avec la DEL pour une tension d’alimentation de 5 V.}
\begin{minipage}[c]{0.65\textwidth}
	\subquestion{Réaliser le montage avec une seule DEL et sa résistance de protection. Après vérification par le professeur, déterminer la puissance de chaque dipôle dans le circuit.}
	\subquestion{Quel est le lien (calcul) entre ces puissances?}
	\subquestion{Réaliser le montage complet du doc. 2 et refaire les mesures}
	\subquestion{Trouver une relation d'égalité entre ces différentes puissances.}
	\subquestion{indiquer comment évolue la puissance du générateur quand on augmente le nombre de dipôles dans le circuit.}
\end{minipage}
% \hspace{0.05\textwidth}
\begin{minipage}[c]{0.35\textwidth}
	\begin{center}
		circuit 2:

		\includegraphics[width=\columnwidth]{c2.png}
	\end{center}
\end{minipage}
% \vspace{-10pt}
% \subquestion{}
% \subquestion{}
% \subquestion{}
% \subquestion{}


\end{document}
