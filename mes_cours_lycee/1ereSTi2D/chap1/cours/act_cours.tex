%%%%%%%%%%%%%%%%%%%%%%%%%%%%%%%%%%%%%%%%%
% Cleese Assignment (For Students)
% LaTeX Template
% Version 2.0 (27/5/2018)
%
% This template originates from:
% http://www.LaTeXTemplates.com
%
% Author:
% Vel (vel@LaTeXTemplates.com)
%
% License:
% CC BY-NC-SA 3.0 (http://creativecommons.org/licenses/by-nc-sa/3.0/)
% 
%%%%%%%%%%%%%%%%%%%%%%%%%%%%%%%%%%%%%%%%%

%----------------------------------------------------------------------------------------
%	PACKAGES AND OTHER DOCUMENT CONFIGURATIONS
%----------------------------------------------------------------------------------------



\documentclass[24pt]{article}

\input{activite.sty} % Include the file specifying the document structure and custom commands
\input{annotate_equations.sty}


\newcommand{\prgrmparun}{
    \mypage{
        \begin{itemize}
            \item Citer les différentes formes d’énergie utilisées dans les
                  domaines de la vie courante, de la production et des
                  services.
            \item Distinguer les formes d’énergie des différentes sources
                  d’énergie associées.
        \end{itemize}
    }
}


\newcommand{\prgrmpardeux}{
    \mypage{
        \begin{itemize}
            \item Énoncer et exploiter la relation entre puissance, énergie
                  et durée.
            \item Évaluer et citer des ordres de grandeur des puissances
                  mises en jeu dans les secteurs de l’énergie, de l’habitat,
                  des transports, des communications, etc.
        \end{itemize}
    }
}

\newcommand{\prgrmpartrois}{
    \mypage{
        \begin{itemize}
            \item Identifier les principales conversions d’énergie :
                  électromécanique, photoélectrique, électrochimique,
                  thermodynamique (conversions réalisées par une
                  machine thermique), etc.
            \item Schématiser une chaîne énergétique ou une conversion
                  d’énergie en distinguant formes d’énergie, sources
                  d’énergie et convertisseurs.
            \item Évaluer ou mesurer une quantité d’énergie transférée,
                  convertie ou stockée.
        \end{itemize}
    }
}

\newcommand{\prgrmparquatre}{
    \mypage{
        \begin{itemize}
            \item Énoncer le principe de conservation de l’énergie pour un
                  système isolé.
            \item Exploiter le principe de conservation de l’énergie pour
                  réaliser un bilan énergétique et calculer un rendement
                  pour une chaîne énergétique ou un convertisseur.
            \item Déterminer le rendement d’une chaîne énergétique ou
                  d’un convertisseur
        \end{itemize}
    }
}


\newcommand{\prgrmparcinq}{
    \mypage{
        \begin{itemize}
            \item Énoncer qu’une ressource d’énergie est qualifiée de
                  « renouvelable » si son renouvellement naturel est
                  assez rapide à l’échelle de temps d’une vie humaine.
        \end{itemize}
    }
}


\newcommand{\mypage}[1]{
    \begin{minipage}[t]{0.6\textwidth}
        {#1}
    \end{minipage}
}

%	\setlength{\arrayrulewidth}{0.5mm}
%	\setlength{\tabcolsep}{18pt}
\newcommand{\programme}{
    \renewcommand{\arraystretch}{2}
    \begin{center}
        \begin{tabular}{@{}|l|l|@{}}
            \multicolumn{2}{c}{Programme : Énergie}                     \\ \midrule

            \begin{minipage}[t]{0.3\textwidth}
                {Savoirs}\end{minipage}         & \mypage{Savoirs-faire}    \\\midrule

            Formes d’énergie.                          & \prgrmparun    \\

            Énergie et puissance.                      & \prgrmpardeux  \\

            \begin{minipage}[t]{0.3\textwidth}
                Les conversions et les chaînes
                énergétiques.

                Stockage de l’énergie.\end{minipage}       & \prgrmpartrois \\

            \begin{minipage}[t]{0.3\textwidth}
                Principe de la conservation de l’énergie.
                Rendement\end{minipage}  & \prgrmparquatre                  \\

            Ressource d’énergie dite « renouvelable ». & \prgrmparcinq  \\

            \bottomrule\end{tabular}
    \end{center}

}




%----------------------------------------------------------------------------------------
%	VARIABLES
%----------------------------------------------------------------------------------------

% Required
\newcommand{\assignmentQuestionName}{Question} % The word to be used as a prefix to question numbers; example alternatives: Problem, Exercise
\newcommand{\assignmentClass}{Physique Chimie --} % Course/class
\newcommand{\assignmentTitle}{Chapitre 1 Mesures et incertitudes} % Assignment title or name
\newcommand{\assignmentAuthorName}{Mme Cercy} 
% \newcommand{\assignmentAuthorName}{Chapitre Y} 
\newcommand{\titre}{Chapitre 1 : Mesures et incertitudes} % titre de l'activité
\newcommand{\titreActivite}{Chapitre 1 : Mesures et incertitudes} % titre de l'activité

%----------------------------------------------------------------------------------------

\begin{document}

\thispagestyle{fancy}
% \cfoot{}

\date{}
\title{\titreActivite}
\maketitle % Print the title page


\tableauProgramme


\assignmentSection{Définitions}
% \vspace{-10pt}


\begin{minipage}[c]{0.45\textwidth}
    \begin{definition}[Grandeurs]
        On appelle grandeur physique, ou simplement grandeur, toute propriété qui peut être mesurée ou calculée.

        voir fascicule "Remise à niveau" pour plus de détail.
    \end{definition}
\end{minipage}
\hspace{ 0pt}
\begin{minipage}[c]{0.55\textwidth}
    \begin{definition}[Valeur]
        Une valeur est un nombre accompagné d'une unité qui représente la quantité d'une grandeur.

        Exemple : $$\eqnmark{grandeurun}{m} = \eqnmarkbox{grandeurdeux}{7.0 kg}$$
        \annotate[yshift=10pt]{right,above}{grandeurun}{la grandeur physique}
        \annotate[yshift=-5pt]{right,below}{grandeurdeux}{la valeur de la grandeur physique}
        % \vspace{30pt}
    \end{definition}
\end{minipage}

\begin{definition}[Unité de base du SI.]
    Les unités de base du Système international sont les sept unités de mesure indépendantes
    du Système international à partir desquelles sont obtenues
    toutes les autres unités, dites unités dérivées.
    % \vspace{5pt}

    \begin{center}
        voir fascicule "Remise à niveau" pour la liste des 7 unités à connaître.
    \end{center}
\end{definition}

\assignmentSection{Savoirs-faire}


\begin{concept}{Valeur de référence.}
    Lors d'une mesure on ne peut pas connaître la valeur vraie d'une grandeur,
    on peut connaître la \textbf{\color{DarkRed} {valeur de référence}}
    qu'on trouve en faisant la mesure avec un appareil précis et exact.
\end{concept}


\begin{concept}{Sources d'erreurs}
    \begin{itemize}
        \item La grandeur que l'on veut mesurer est appelée le mesurande, noté M.
        \item Le mesurage est le processus consistant à obtenir expérimentalement une
              ou plusieurs valeurs, que l'on peut attribuer à une grandeur.
        \item La valeur vraie du mesurande est une valeur théorique que l'on
              obtiendrait si le mesurage était parfait.
    \end{itemize}

    Lors de la mesure d'une grandeur physique,
    l'erreur de mesure est la différence entre la
    valeur mesurée et la valeur vraie.
    On ne peut pas connaître la valeur vraie, on peut connaître la
    \textbf{valeur de référence}.

    Dans des mesurages répétés, on distingue deux composantes de l'erreur de mesure :
    \begin{itemize}
        \item l'erreur systématique qui est constante ou varie de manière prévisible.
        \item l'erreur aléatoire qui varie de façon imprévisible.
    \end{itemize}
    L'incertitude de mesure $U(M)$ est un paramètre, associé au résultat d'une mesure, qui
    caractérise la dispersion des valeurs qui pourraient être attribuées à la valeur mesurée.

    Le résultat est un intervalle des valeurs probables de la valeur mesurée avec l'unité appropriée.
    $$M =( m±U(M) ) unit\acute{e}$$

\end{concept}

\begin{concept}{Incertitude-type A}
    Les incertitudes de types A sont utiles quand un grand nombre de mesures
    sont réalisées (dans les mêmes conditions).
    C'est souvent le cas lorsque l'on ne dispose que de peu d’informations
    sur les sources d’erreurs, puisque celles-ci n’entrent pas en compte dans le calcul.

    Cela permet d’ignorer l’effet des erreurs aléatoires,
    dont on estime qu’elles se compensent en moyenne, mais pas les erreurs systématiques,
    dont il faut toujours tenir compte.

    Pour calculer l'incertitude de type A on utilise la formule suivante:
    \vspace{60pt}
    $$
        \eqnmarkbox[blue]{ux}{\text{U}(X)} =
        \eqnmark{k}{\mathrm{k}}
        \dfrac{\eqnmark{sigma}{\sigma} }{\sqrt{\eqnmark{n}{n}} }$$

    \annotate[yshift=5pt]{left,above}{ux}{Incertitude}
    \annotate[yshift=25pt]{above}{k}{
        \begin{minipage}[c]{100pt}
            un coefficient et qui dépend de n et du niveau
            de confiance sur la mesure (souvent 95 \% auquel cas, $k = 2$).
        \end{minipage}

    }
    \annotate[yshift=10pt,xshift=100pt]{right}{sigma}{\begin{minipage}[c]{120pt}
            l’écart-type de la série de mesures représente
            l’étalement des mesures autour de la moyenne, peut être obtenu facilement avec un tableur
            \vspace{5pt}
        \end{minipage}}
    \annotate[xshift=30pt]{below}{n}{le nombre de mesures}


    où $σ$ est défini tel que :

    $$\sigma = \sqrt{\frac{1}{n}\sum_{i=1}^n (\eqnmark{mi}{m_i} - \eqnmark{moy}{\overline{m}})^2}$$
    \annotate[xshift=-80pt]{below,left}{mi}{la valeur de la i-ème mesure}
    \annotate[xshift=30pt]{below,right}{moy}{la moyenne de la valeur des mesures}
    \vspace{5pt}

    \textit{voir le problème 2 de "Remise à niveau"}

\end{concept}



\begin{concept}{Justesse et fidélité}
    Le résultat d'une mesure est
    \begin{itemize}
        \item Juste si quand on répète la mesure, l'écart entre la valeur moyenne trouvée
              et la valeur de référence est faible.
        \item Fidèle si quand on répète la mesure, l'écart entre les mesures est faible.
    \end{itemize}
    \begin{center}
        \includegraphics[width=0.4\columnwidth]{juste_fidele.jpeg}
    \end{center}

\end{concept}
\begin{concept}{Écriture d'un résultat}
    \begin{enumerate}
        \item Compter les chiffres significatifs

              \begin{itemize}
                  \item Les chiffres significatifs sont tous les chiffres \textbf{sauf les zéro à gauche}
                  \item Les puissances de dix ne comptent pas pour les chiffres significatifs.
              \end{itemize}

        \item Si le calcul est une addition ou une soustraction, le résultat doit avoir autant de chiffres après la virgule que le terme qui en comporte le moins.

              Exemple :
              $$\eqnmark{addun}{8.0} + \eqnmark{adddeux}{1.06} = \eqnmark{addres}{9.06} = 9.1$$

              \annotate[xshift=-20pt,yshift=20pt]{below,left}{addun}{
                  \begin{minipage}[c]{100pt}
                      Le terme qui a le moins de chiffre après la virgule est 8.0 qui en a 1.
                      \vspace{10pt}
                  \end{minipage}
              }
              \annotate[yshift=20pt,xshift=50pt]{below}{addres}{
                  \begin{minipage}[c]{100pt}
                      Le résultat doit donc avoir un chiffre après la virgule, il faut arrondir.
                      \vspace{10pt}
                  \end{minipage}

              }
              %   \annotate[yshift=-5pt]{below}{adddeux}{}


        \item Si le calcul est une multiplication ou une division, le résultat final a autant de chiffre significatifs que le terme qui a le moins de chiffres significatifs.

              Exemple :
              $$
                  \eqnmark{huit}{8.0}
                  \times
                  \eqnmark{deux}{2}
                  =
                  \eqnmark{seize}{16}
                  = 20$$
              \annotate[yshift=-5pt]{below,left}{huit}{deux CS}
              \annotate[yshift=-5pt]{below}{deux}{un CS}
              \annotate[yshift=5pt]{above}{seize}{Donc le résultat aura un CS: 16 a deux CS, il faut donc arrondir}

              %           \item 
              %           \item .
              %       \end{itemize}

              %       9.06 a deux chiffres après la virgule il faut donc arrondir à 9.1
              %   \end{minipage}

        \item Il ne faut pas arrondir plusieurs fois pour écrire un résultat mais faire tout le calcul d'un coup.

    \end{enumerate}
\end{concept}

% \begin{concept}{Sources d'erreur}

% \end{concept}


% \assignmentSection{Entraînements simple}




% \clearpage


\title{Activité}
\maketitle

%----------------------------------------------------------------------------------------
%	QUESTION 
%----------------------------------------------------------------------------------------

\begin{center}
    \large \begin{list}{$\bullet$}{}
        \item \textbf{Répondre aux les questions suivantes sur une feuille (qui sera ramassé) avec nom et prénom . }
        \item \textbf{Répondre par des phrases.}
        \item \textbf{Écrire toute les justifications nécessaires.}
        \item \textbf{Rédiger proprement.}
    \end{list}
\end{center}


%----------------------------------------------------------------------------------------
%	QUESTION 
%----------------------------------------------------------------------------------------

\begin{question}
    \questiontext{
        Questions de cours
    }
\end{question}
\subquestion{Quelles sont les 7 unités de base du Système International?}

%----------------------------------------------------------------------------------------
%	QUESTION 
%----------------------------------------------------------------------------------------

\begin{question}
    \questiontext{
        Identifier les sources d'erreurs dans la mesure suivante:
    }
\end{question}
%	\answerbox{}

\begin{minipage}[c]{0.45\textwidth}
    \begin{question}
        \questiontext{
            On mesure avec une balance :
            \begin{center}
                $m=(475.21 ± 0.03) μg$
            \end{center}
        }
    \end{question}
    \subquestion{Quelle est la grandeur mesurée?}
    \subquestion{Quelle est la valeur mesurée?}
    \subquestion{Quelle est l'unité de la valeur mesurée?}
    \subquestion{Que vaut l'incertitude?}
    \subquestion{Combien y a-t-il de chiffre significatif sur la mesurage ? Sur l'incertitude ?}
\end{minipage}
\hspace{0.1\textwidth}
\begin{minipage}[c]{0.45\textwidth}
    \begin{question}
        \questiontext{
            Pour chacun des mesurages suivants:
            \begin{list}{$\bullet$}{}
                \item $V= (100.0 ± 0.5) mL$
                \item $t = (60.00 ± 0.4 )s$
                \item $m = (3.56 ± 0.0584)g $
                \item $L = (10 ± 0.5)cm$
            \end{list}
        }
    \end{question}
    \subquestion{Rectifier si nécessaire le mesurage}
    \subquestion{Écrire le résultat modifié en notation scientifique.}
\end{minipage}


%----------------------------------------------------------------------------------------
%	QUESTION 
%----------------------------------------------------------------------------------------

% \begin{question}
%     \questiontext{
%         valeur de référence
%     }
% \end{question}
%	\answerbox{}

\begin{question}
    \questiontext{
        Écrire correctement le résultat des mesurages suivants
    }
\end{question}
\subquestion{Avec une règle, on mesure $l = 90,5 cm$. L'incertitude-type de lecture vaut $u_{lecture}= 1 cm$ .}
\subquestion{
    Avec une balance, on pèse $m = 0,896 g$. L'incertitude-type de résolution vaut $u_{res}=0,02 g$ .
}
\subquestion{On mesure une tension $U = 12,05 V$. L'incertitude-type de précision vaut $u_{pre} =0,01V$}
%	\answerbox{}

\begin{question}
    \questiontext{
        Voici les caractéristiques d'une balance.

        \begin{itemize}
            \item Portée max : 220 g
            \item Portée min : 0,02 g
            \item Résolution : 0,1 mg
            \item Écart maximal toléré : 1 mg
        \end{itemize}

        \begin{center}
            \begin{tabular}{@{}|l|llllll|@{}} \toprule
                Mesure n°    & 1       & 2       & 3       & 4       & 5       & 6       \\
                Valeur ($Ω$) & 50,0000 & 49,9997 & 49,9996 & 50,0004 & 50,0001 & 49,9999 \\\bottomrule
                % \multicolumn{2}{c}{Item} \\ \cmidrule(r){1-2}
                % Animal & Description & Price (\$)\\ \midrule
            \end{tabular}
        \end{center}
    }
    \subquestion{
        Calculer l'écart type $σ$ pour ces mesures.
    }
    \subquestion{
        Comparer $σ$ à l'écart maximal toléré. Conclure sur la fidélité de la balance.
    }
    \subquestion{
        Calculer l'incertitude type A $U(r)$
    }
    % \subquestion{
    %     Présenter les résultats de ces deux mesurages sous la forme qui convient en utilisant
    %     l'écriture scientifique.
    % }
\end{question}





\end{document}
