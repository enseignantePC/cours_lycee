%%%%%%%%%%%%%%%%%%%%%%%%%%%%%%%%%%%%%%%%%
% Cleese Assignment (For Students)
% LaTeX Template
% Version 2.0 (27/5/2018)
%
% This template originates from:
% http://www.LaTeXTemplates.com
%
% Author:
% Vel (vel@LaTeXTemplates.com)
%
% License:
% CC BY-NC-SA 3.0 (http://creativecommons.org/licenses/by-nc-sa/3.0/)
% 
%%%%%%%%%%%%%%%%%%%%%%%%%%%%%%%%%%%%%%%%%

%----------------------------------------------------------------------------------------
%	PACKAGES AND OTHER DOCUMENT CONFIGURATIONS
%----------------------------------------------------------------------------------------



\documentclass[24pt]{article}

\input{activite.sty} % Include the file specifying the document structure and custom commands
\input{annotate_equations.sty}
% 

\newcommand{\prgrmparun}{
    \mypage{
        \begin{itemize}
            \item Citer les différentes formes d’énergie utilisées dans les
                  domaines de la vie courante, de la production et des
                  services.
            \item Distinguer les formes d’énergie des différentes sources
                  d’énergie associées.
        \end{itemize}
    }
}


\newcommand{\prgrmpardeux}{
    \mypage{
        \begin{itemize}
            \item Énoncer et exploiter la relation entre puissance, énergie
                  et durée.
            \item Évaluer et citer des ordres de grandeur des puissances
                  mises en jeu dans les secteurs de l’énergie, de l’habitat,
                  des transports, des communications, etc.
        \end{itemize}
    }
}

\newcommand{\prgrmpartrois}{
    \mypage{
        \begin{itemize}
            \item Identifier les principales conversions d’énergie :
                  électromécanique, photoélectrique, électrochimique,
                  thermodynamique (conversions réalisées par une
                  machine thermique), etc.
            \item Schématiser une chaîne énergétique ou une conversion
                  d’énergie en distinguant formes d’énergie, sources
                  d’énergie et convertisseurs.
            \item Évaluer ou mesurer une quantité d’énergie transférée,
                  convertie ou stockée.
        \end{itemize}
    }
}

\newcommand{\prgrmparquatre}{
    \mypage{
        \begin{itemize}
            \item Énoncer le principe de conservation de l’énergie pour un
                  système isolé.
            \item Exploiter le principe de conservation de l’énergie pour
                  réaliser un bilan énergétique et calculer un rendement
                  pour une chaîne énergétique ou un convertisseur.
            \item Déterminer le rendement d’une chaîne énergétique ou
                  d’un convertisseur
        \end{itemize}
    }
}


\newcommand{\prgrmparcinq}{
    \mypage{
        \begin{itemize}
            \item Énoncer qu’une ressource d’énergie est qualifiée de
                  « renouvelable » si son renouvellement naturel est
                  assez rapide à l’échelle de temps d’une vie humaine.
        \end{itemize}
    }
}


\newcommand{\mypage}[1]{
    \begin{minipage}[t]{0.6\textwidth}
        {#1}
    \end{minipage}
}

%	\setlength{\arrayrulewidth}{0.5mm}
%	\setlength{\tabcolsep}{18pt}
\newcommand{\programme}{
    \renewcommand{\arraystretch}{2}
    \begin{center}
        \begin{tabular}{@{}|l|l|@{}}
            \multicolumn{2}{c}{Programme : Énergie}                     \\ \midrule

            \begin{minipage}[t]{0.3\textwidth}
                {Savoirs}\end{minipage}         & \mypage{Savoirs-faire}    \\\midrule

            Formes d’énergie.                          & \prgrmparun    \\

            Énergie et puissance.                      & \prgrmpardeux  \\

            \begin{minipage}[t]{0.3\textwidth}
                Les conversions et les chaînes
                énergétiques.

                Stockage de l’énergie.\end{minipage}       & \prgrmpartrois \\

            \begin{minipage}[t]{0.3\textwidth}
                Principe de la conservation de l’énergie.
                Rendement\end{minipage}  & \prgrmparquatre                  \\

            Ressource d’énergie dite « renouvelable ». & \prgrmparcinq  \\

            \bottomrule\end{tabular}
    \end{center}

}




%----------------------------------------------------------------------------------------
%	VARIABLES
%----------------------------------------------------------------------------------------

% Required
\newcommand{\assignmentQuestionName}{Question} % The word to be used as a prefix to question numbers; example alternatives: Problem, Exercise
\newcommand{\assignmentClass}{Physique Chimie --} % Course/class
\newcommand{\assignmentTitle}{Chapitre 1: Les éléments chimiques} % Assignment title or name
\newcommand{\assignmentAuthorName}{Mme Cercy} 
% \newcommand{\assignmentAuthorName}{Chapitre Y} 
% \newcommand{\titre}{Activité 1 : Les éléments chimiques dans les étoiles} % titre de l'activité
\newcommand{\titreActivite}{\huge Activité 1 : Les éléments chimiques dans les étoiles} % titre de l'activité

%----------------------------------------------------------------------------------------

\begin{document}

\thispagestyle{fancy}
% \cfoot{}

\date{}
\title{\titreActivite}
\maketitle % Print the title page

\begin{center}
    Les astrophysiciens identifient les éléments chimiques présents dans les étoiles en analysant leurs spectres lumineux.

    Comment peut-on déterminer la composition des étoiles à partir de leur spectre lumineux ?
\end{center}


\begin{minipage}[c]{0.45\textwidth}
    \begin{definition}[Spectre d'émission]
        Spectre d’une lumière directement émise par une source.
    \end{definition}
\end{minipage}
\hspace{0.1\textwidth}
\begin{minipage}[c]{0.45\textwidth}
    \begin{definition}[Spectre d'absorption]
        Spectre d’une lumière blanche ayant traversé une substance (gaz, filtre, solution, etc.).

    \end{definition}
\end{minipage}

\begin{documentpeda}{Spectre du Soleil}
    Le spectre d’une étoile est la superposition d’un spectre
    d'émission d’origine thermique (fond coloré) et de raies d’absorption (spectre d’absorption, raies noires).
    La longueur d’onde de ces raies permet d’identifier les éléments chimiques présents dans l’étoile.
    Dans le cas du Soleil, ces raies sont appelées « raies de Fraunhofer »,
    du nom du physicien et opticien allemand qui les a observées pour la première fois au XIXe siècle.
    Les raies notées de A à H sont celles qu’il a identifiées en 1814.

    \begin{center}
        \includegraphics[width=0.7\columnwidth]{spectre.png}
    \end{center}
\end{documentpeda}


\begin{documentpeda}{Composition chimique de la photosphère}

    \hspace{20pt}
    \begin{minipage}[c]{0.45\textwidth}
        Une étoile est une masse gazeuse dans laquelle on trouve principalement de l’hydrogène et de l’hélium.
        Elle peut contenir aussi d’autres éléments chimiques, dont la nature et l’abondance dépendent de l’étoile.
        Tous les éléments chimiques de numéro atomique supérieur à 2 sont considérés comme des
        « éléments lourds » en astrophysique.
    \end{minipage}
    \hspace{20pt}
    \begin{minipage}[c]{0.45\textwidth}
        \begin{center}
            %	\setlength{\arrayrulewidth}{0.5mm}
            \setlength{\tabcolsep}{18pt}
            \renewcommand{\arraystretch}{1.5}
            \begin{tabular}{@{}|l|c|@{}} \toprule
                Hydrogène (Z = 1) & 73,46 \% \\
                Hélium (Z = 2)    & 24,85 \% \\
                Oxygène (Z = 8)   & 0,77  \% \\
                Carbone (Z = 6)   & 0,29  \% \\
                Fer (Z = 26)      & 0,16  \% \\
                Néon (Z = 10)     & 0,12  \% \\
                \bottomrule
            \end{tabular}
        \end{center}
    \end{minipage}





\end{documentpeda}

\assignmentSection{Questions}
% \vspace{-10pt}

%----------------------------------------------------------------------------------------
%	QUESTION 1
%----------------------------------------------------------------------------------------

\begin{question}
    \questiontext{
        L’atome d’hydrogène est caractérisé par plusieurs raies lumineuses. Une des raies a une longueur d’onde de 656 nm. Peut-on identifier cette raie sur le spectre du document 1 ?
    }
\end{question}
\answerbox{2}

%----------------------------------------------------------------------------------------
%	QUESTION 2
%----------------------------------------------------------------------------------------

\begin{question}
    \questiontext{
        À l'aide du document 2, Tracez un diagramme circulaire donnant la composition chimique en pourcentage massique du Soleil.
    }
\end{question}
%	\answerbox{}


%----------------------------------------------------------------------------------------
%	QUESTION 3
%----------------------------------------------------------------------------------------

\begin{question}
    \questiontext{
        Expliquez comment identifier les éléments chimiques présents dans une étoile.
    }
\end{question}
%	\answerbox{}

\end{document}
