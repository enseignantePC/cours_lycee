%%%%%%%%%%%%%%%%%%%%%%%%%%%%%%%%%%%%%%%%%
% Cleese Assignment (For Students)
% LaTeX Template
% Version 2.0 (27/5/2018)
%
% This template originates from:
% http://www.LaTeXTemplates.com
%
% Author:
% Vel (vel@LaTeXTemplates.com)
%
% License:
% CC BY-NC-SA 3.0 (http://creativecommons.org/licenses/by-nc-sa/3.0/)
% 
%%%%%%%%%%%%%%%%%%%%%%%%%%%%%%%%%%%%%%%%%

%----------------------------------------------------------------------------------------
%	PACKAGES AND OTHER DOCUMENT CONFIGURATIONS
%----------------------------------------------------------------------------------------



\documentclass[24pt]{article}

\input{activite.sty} % Include the file specifying the document structure and custom commands
\input{annotate_equations.sty}


%----------------------------------------------------------------------------------------
%	VARIABLES
%----------------------------------------------------------------------------------------

% Required
\newcommand{\assignmentQuestionName}{Question} % The word to be used as a prefix to question numbers; example alternatives: Problem, Exercise
\newcommand{\assignmentClass}{Physique Chimie --} % Course/class
\newcommand{\assignmentTitle}{Chapitre 1: Les éléments chimiques} % Assignment title or name
\newcommand{\assignmentAuthorName}{Mme Cercy} 
% \newcommand{\assignmentAuthorName}{Chapitre Y} 
% \newcommand{\titre}{Activité 1 : Les éléments chimiques dans les étoiles} % titre de l'activité
\newcommand{\titreActivite}{\huge Act 2 : La radioactivité, un phénomène naturel} % titre de l'activité

%----------------------------------------------------------------------------------------

\begin{document}

\thispagestyle{fancy}
% \cfoot{}

\date{}
\title{\titreActivite}
\maketitle % Print the title page


\begin{documentpeda}{Une découverte fortuite}
    En 1896, Henri Becquerel étudie les propriétés de fluorescence des sels d’uranium en
    les exposant aux rayons solaires, puis en les déposant sur une plaque photographique.
    Après quelques minutes, la plaque est impressionnée comme si elle avait été exposée à la lumière.
    Henri Becquerel pense
    que ce sont les rayonnements absorbés par l’uranium qui sont réémis sous forme de rayons X vers la plaque.
    C’est par hasard qu’il découvre que si les sels restent plusieurs jours dans un tiroir, une image apparaît
    également sur une plaque photographique à proximité ! Sa théorie sur la fluorescence des sels d’uranium
    est remise en cause. L’uranium émet des rayonnements de façon « naturelle ».
\end{documentpeda}

\begin{minipage}[c]{0.45\textwidth}
    \begin{documentpeda}{La loi de décroissance radioactive}
        Dans un échantillon contenant au départ $N_0$ atomes radioactifs,
        le nombre de noyaux décroît de telle sorte que le nombre N
        de noyaux est divisé par deux au bout d’une durée appelée « demi-vie » notée $t_{\frac{1}{2}}$
        et qui dépend de la nature du noyau. Par exemple, la demi-vie du carbone 14 vaut 5 730 ans.
        \begin{center}
            \includegraphics[width=\columnwidth]{doc2.png}
        \end{center}
    \end{documentpeda}
\end{minipage}
\hspace{0.1\textwidth}
\begin{minipage}[c]{0.45\textwidth}
    \begin{documentpeda}{La désintégration du carbone 14}
        Le carbone 14 $(Z=6)$ est un noyau radioactif instable, qui se désintègre en libérant un électron et
        en se transformant en un autre noyau,
        l’azote 14 $(Z=7)$. Dans le noyau, un neutron se transforme en proton en éjectant un électron.
        Le noyau ainsi formé se désexcite ensuite en émettant un rayonnement gamma $(γ)$.

        \begin{center}
            \includegraphics[width=\columnwidth]{doc3.png}
        \end{center}
    \end{documentpeda}
\end{minipage}

\begin{minipage}[c]{0.45\textwidth}
    \begin{definition}[Fluorescence]
        Propriété d’un corps qui, après avoir été éclairé, émet de la lumière.
    \end{definition}
\end{minipage}
\hspace{0.1\textwidth}
\begin{minipage}[c]{0.45\textwidth}
    \begin{definition}[Noyau radioactif]
        Un noyau instable est radioactif, il va se désintégrer inéluctablement à un instant qui n’est pas prévisible.
    \end{definition}
\end{minipage}

\assignmentSection{Questions}

%----------------------------------------------------------------------------------------
%	QUESTION 
%----------------------------------------------------------------------------------------

\begin{question}
    \questiontext{
        Répondre à l'aide du document 1.
        Quelles observations de Becquerel lui ont permis d’affirmer que la radioactivité est un phénomène naturel ?
    }
\end{question}
\answerbox{2}

%----------------------------------------------------------------------------------------
%	
%----------------------------------------------------------------------------------------

\begin{question}
    \questiontext{
        Répondre à l'aide du document 2.
        Combien de temps faut-il pour que la population en carbone 14 d’un échantillon soit divisée par 2 ? Et par 4 ?
    }
\end{question}
% \answerbox{1}

%----------------------------------------------------------------------------------------
%	QUESTION 
%----------------------------------------------------------------------------------------

\begin{question}
    \questiontext{
        Répondre à l'aide du document 3.
    }
\end{question}
%	\answerbox{}
\subquestion{Donnez la composition du noyau de l’atome de carbone 14}
\subquestion{
    Identifiez la modification qui a eu lieu dans le noyau de l’atome de carbone 14 au cours de
    sa désintégration et écrivez l’équation de la désintégration.
}

\end{document}

%----------------------------------------------------------------------------------------
%	QUESTION 
%----------------------------------------------------------------------------------------

\begin{question}
    \questiontext{
        Expliquez le phénomène de radioactivité à travers l’exemple du carbone 14.
    }
\end{question}
\answerbox{4}