%%%%%%%%%%%%%%%%%%%%%%%%%%%%%%%%%%%%%%%%%
% Cleese Assignment (For Students)
% LaTeX Template
% Version 2.0 (27/5/2018)
%
% This template originates from:
% http://www.LaTeXTemplates.com
%
% Author:
% Vel (vel@LaTeXTemplates.com)
%
% License:
% CC BY-NC-SA 3.0 (http://creativecommons.org/licenses/by-nc-sa/3.0/)
% 
%%%%%%%%%%%%%%%%%%%%%%%%%%%%%%%%%%%%%%%%%

%----------------------------------------------------------------------------------------
%	PACKAGES AND OTHER DOCUMENT CONFIGURATIONS
%----------------------------------------------------------------------------------------



\documentclass[24pt]{article}

\input{activite.sty} % Include the file specifying the document structure and custom commands
\input{annotate_equations.sty}


\newcommand{\prgrmparun}{
    \mypage{
        \begin{itemize}
            \item Citer les différentes formes d’énergie utilisées dans les
                  domaines de la vie courante, de la production et des
                  services.
            \item Distinguer les formes d’énergie des différentes sources
                  d’énergie associées.
        \end{itemize}
    }
}


\newcommand{\prgrmpardeux}{
    \mypage{
        \begin{itemize}
            \item Énoncer et exploiter la relation entre puissance, énergie
                  et durée.
            \item Évaluer et citer des ordres de grandeur des puissances
                  mises en jeu dans les secteurs de l’énergie, de l’habitat,
                  des transports, des communications, etc.
        \end{itemize}
    }
}

\newcommand{\prgrmpartrois}{
    \mypage{
        \begin{itemize}
            \item Identifier les principales conversions d’énergie :
                  électromécanique, photoélectrique, électrochimique,
                  thermodynamique (conversions réalisées par une
                  machine thermique), etc.
            \item Schématiser une chaîne énergétique ou une conversion
                  d’énergie en distinguant formes d’énergie, sources
                  d’énergie et convertisseurs.
            \item Évaluer ou mesurer une quantité d’énergie transférée,
                  convertie ou stockée.
        \end{itemize}
    }
}

\newcommand{\prgrmparquatre}{
    \mypage{
        \begin{itemize}
            \item Énoncer le principe de conservation de l’énergie pour un
                  système isolé.
            \item Exploiter le principe de conservation de l’énergie pour
                  réaliser un bilan énergétique et calculer un rendement
                  pour une chaîne énergétique ou un convertisseur.
            \item Déterminer le rendement d’une chaîne énergétique ou
                  d’un convertisseur
        \end{itemize}
    }
}


\newcommand{\prgrmparcinq}{
    \mypage{
        \begin{itemize}
            \item Énoncer qu’une ressource d’énergie est qualifiée de
                  « renouvelable » si son renouvellement naturel est
                  assez rapide à l’échelle de temps d’une vie humaine.
        \end{itemize}
    }
}


\newcommand{\mypage}[1]{
    \begin{minipage}[t]{0.6\textwidth}
        {#1}
    \end{minipage}
}

%	\setlength{\arrayrulewidth}{0.5mm}
%	\setlength{\tabcolsep}{18pt}
\newcommand{\programme}{
    \renewcommand{\arraystretch}{2}
    \begin{center}
        \begin{tabular}{@{}|l|l|@{}}
            \multicolumn{2}{c}{Programme : Énergie}                     \\ \midrule

            \begin{minipage}[t]{0.3\textwidth}
                {Savoirs}\end{minipage}         & \mypage{Savoirs-faire}    \\\midrule

            Formes d’énergie.                          & \prgrmparun    \\

            Énergie et puissance.                      & \prgrmpardeux  \\

            \begin{minipage}[t]{0.3\textwidth}
                Les conversions et les chaînes
                énergétiques.

                Stockage de l’énergie.\end{minipage}       & \prgrmpartrois \\

            \begin{minipage}[t]{0.3\textwidth}
                Principe de la conservation de l’énergie.
                Rendement\end{minipage}  & \prgrmparquatre                  \\

            Ressource d’énergie dite « renouvelable ». & \prgrmparcinq  \\

            \bottomrule\end{tabular}
    \end{center}

}



%----------------------------------------------------------------------------------------
%	VARIABLES
%----------------------------------------------------------------------------------------

% % Required
\newcommand{\assignmentQuestionName}{Exercice} % The word to be used as a prefix to question numbers; example alternatives: Problem, Exercise
\newcommand{\assignmentClass}{Physique Chimie ---} % Course/class
\newcommand{\assignmentTitle}{Chapitre 1: Les éléments chimiques} % Assignment title or name
\newcommand{\assignmentAuthorName}{Mme Cercy} 
% \newcommand{\assignmentAuthorName}{Chapitre 1} 
% % \newcommand{\titre}{Activité 1 : Les éléments chimiques dans les étoiles} % titre de l'activité
% \newcommand{\titreActivite}{\huge Act 2 : La radioactivité, un phénomène naturel} % titre de l'activité

%----------------------------------------------------------------------------------------

\begin{document}
% \thispagestyle{fancy}
% \cfoot{}
\chead{2022-2023}
\date{}
\title{\huge Chapitre 1: Les éléments chimiques}
\maketitle % Print the title page

\programme

\assignmentSection{Cours}


\begin{concept}{Désintégration des noyaux radioactifs }
    \begin{center}
        \textbf{La radioactivité}
    \end{center}
    \begin{list}{$\bullet$}{}
        \item
              \begin{minipage}[c]{0.6\textwidth}
                  Certains noyaux sont instables : on dit qu’ils sont radioactifs.
                  La radioactivité est un phénomène naturel, qui résulte de la transmutation d’un noyau
                  en un autre.

                  Ainsi, les désintégrations successives ont contribué à la
                  formation des 94 éléments chimiques que l’on trouve sur Terre.

                  Un noyau radioactif peut spontanément se désintégrer en émettant soit une particule $α$ (noyau d’hélium),
                  soit un électron, soit un positon.
              \end{minipage}
              \hspace{0.05\textwidth}
              \begin{minipage}[c]{0.25\textwidth}
                  \begin{center}
                      \includegraphics[width=\columnwidth]{nuclear3.png}
                  \end{center}
              \end{minipage}

              \vspace{10pt}
        \item La radioactivité est aléatoire, inéluctable, spontanée et indépendante de
              la substance dans laquelle le noyau radioactif se trouve.
    \end{list}

    \begin{center}
        \textbf{Évolution du nombre de noyaux et demi-vie }
    \end{center}

    \begin{minipage}[c]{0.7\textwidth}
        La population de noyaux d’un échantillon décroît au cours du temps,
        elle est divisée par deux au bout d’une durée appelée « demi-vie ».

        La courbe à côté représente l’évolution du nombre N de noyaux radioactifs en fonction
        du temps t. La désintégration suit une loi mathématique de décroissance.
        $N_0$ représente le nombre de noyaux radioactifs à l’instant $t_0$ (origine des dates) ;
        $t_{\frac{1}{2}}$ représente la demi-vie.
    \end{minipage}
    \hspace{0.05\textwidth}
    \begin{minipage}[c]{0.2\textwidth}
        \begin{center}
            \includegraphics[width=1.2\columnwidth]{nuclear4.png}
        \end{center}
    \end{minipage}
\end{concept}


\begin{concept}{Les éléments chimiques dans l’Univers}
    \begin{minipage}[c]{0.5\textwidth}
        \begin{center}
            \textbf{La composition chimique de l’Univers}
        \end{center}
        \begin{list}{$\bullet$}{}
            \item L’Univers est formé de 118 éléments chimiques différents.
                  L’hydrogène $_1^1$H est l’élément chimique le plus abondant :
                  il représente à lui seul près de 75 \%
                  des atomes présents dans l’Univers.

            \item Sur Terre, on a observé 94 éléments chimiques
                  à l’état naturel, 24 autres ont été créés artificiellement.

        \end{list}
    \end{minipage}
    \hspace{0.05\textwidth}
    \begin{minipage}[c]{0.4\textwidth}
        \begin{center}
            \textbf{La répartition des éléments chimiques}
        \end{center}

        Les éléments sont répartis de manière inégale dans l’Univers :
        on trouve majoritairement de
        l’hydrogène et de l’hélium dans les étoiles, tandis que la Terre
        est formée principalement d’oxygène et de silicium.

    \end{minipage}
\end{concept}




\begin{concept}{Les réactions nucléaires}
    \begin{center}
        \textbf{La fusion, à l’origine de la synthèse des noyaux}
    \end{center}

    \begin{list}{$\bullet$}{}
        \item
              \begin{minipage}[c]{0.6\textwidth}
                  Selon les théories les plus récentes, les premiers atomes ont été formés quelques minutes
                  après le « Big Bang ». L’Univers était alors extrêmement chaud ($10^9$ K) et dense,
                  les particules élémentaires se sont agglomérées pour former des noyaux d’hydrogène ($^1_1$H), de
                  deutérium ($^2_1$H ou $^2_1$D) et d’hélium et de lithium.

                  Cette réaction nucléaire est appelée fusion nucléaire.
              \end{minipage}
              \hspace{0.05\textwidth}
              \begin{minipage}[c]{0.3\textwidth}
                  Rappel : Tout les éléments se notent:
                  \vspace{10pt}
                  $$^{\eqnmark{nucleon}{N}}_{\eqnmark{proton}{P}}\eqnmark{elem}{X}$$
                  \annotate[yshift=5pt]{right}{elem}{Un atome}
                  \annotate[yshift=5pt]{left,above}{nucleon}{nombre de nucleon}
                  \annotate[]{left,below}{proton}{nombre de protons}
              \end{minipage}

              \vspace{10pt}
        \item
              \begin{minipage}[c]{0.6\textwidth}
                  Au cours d’une réaction de fusion, des noyaux légers forment un noyau plus lourd en éjectant
                  une particule et en libérant de l’énergie. L’équation de la fusion du deutérium avec le
                  tritium s’écrit : $$^2_1H + ^3_1H \rightarrow ^4_2He + ^1_0n $$

                  La réaction libère une énergie de 3,5 MeV.
              \end{minipage}
              \hspace{0.05\textwidth}
              \begin{minipage}[c]{0.3\textwidth}
                  \begin{center}
                      \includegraphics[width=0.8\columnwidth]{nuclear1.png}
                  \end{center}
              \end{minipage}
    \end{list}

    \begin{center}
        \textbf{Les réactions nucléaires au cœur des étoiles}
    \end{center}

    \begin{list}{$\bullet$}{}
        \item Les autres éléments sont formés au sein des étoiles, formées par accrétion des
              atomes créés lors du Big Bang.
              Les noyaux légers fusionnent et produisent des noyaux plus lourds.
              On y trouve ainsi plusieurs éléments comme l’oxygène (Z=8), le carbone (Z=6),
              mais aussi des noyaux plus lourds comme le fer (Z=26).

        \item Sous l’impact de neutrons ou d’autres particules légères, certains noyaux se
              cassent : c’est la fission.

              % \item Ces éléments chimiques sont dispersés à la fin de la vie de l’étoile.
    \end{list}

    \begin{center}
        \textbf{La fission nucléaire}
    \end{center}

    Au cours d’une réaction de fission, des noyaux lourds se cassent en deux noyaux
    plus légers sous l’impact d’un neutron ou d'un proton.
    La réaction s’accompagne de l’éjection d’une particule et libère de l’énergie.
    L’équation de la fission de l’azote bombardé par un proton s’écrit :
    $$^{15}N + ^1H \rightarrow ^{12}C + ^4He$$
    La réaction libère une énergie de 4,96 MeV.

    \begin{center}
        \includegraphics[width=0.4\columnwidth]{nuclear2.png}
    \end{center}
\end{concept}

\assignmentSection{Exercices}


%----------------------------------------------------------------------------------------
%	QUESTION 
%----------------------------------------------------------------------------------------

\begin{question}
    \questiontext{
        L'Univers est constitué de 118 éléments chimiques. Seuls 94 ont été observés sur Terre, les autres étant instables.

        %	\setlength{\arrayrulewidth}{0.5mm}
        \setlength{\tabcolsep}{8pt}
        % \renewcommand{\arraystretch}{1.5}
        \begin{center}
            \begin{tabular}{@{}|l|l|l|l|l|l|l|l|l|l|l|@{}} \toprule
                Numéro atomique $Z$ & 1       & 2       & 6     & 7   & 8      & 10    & 12  & 14  & 16  & 26    \\
                Élément             & H       & He      & C     & N   & O      & Ne    & Mg  & Si  & S   & Fe    \\
                Concentration (ppm) & 739 000 & 240 000 & 4 600 & 960 & 10 400 & 1 340 & 580 & 650 & 440 & 1 090 \\
                Pourcentages        &         &         &       &     &        &       &     &     &     &       \\
                \bottomrule\end{tabular}
        \end{center}

        Ce tableau donne, en ppm (parties par million), la répartition des dix éléments les plus présents dans notre Galaxie.
    }
\end{question}
\subquestion{ Convertissez en pourcentage les valeurs données en ppm (parties par million) et compléter le tableau.}
\subquestion{Représentez sous la forme d'un diagramme en bâtons l'abondance des éléments chimiques les plus fréquents.}

%----------------------------------------------------------------------------------------
%	QUESTION 
%----------------------------------------------------------------------------------------

\begin{question}
    \questiontext{
        Pour remplacer les centrales nucléaires à fission, qui produisent de nombreux déchets radioactifs et posent des problèmes environnementaux, la recherche s'oriente vers les réactions de fusion entre le deutérium et le tritium. Les nouvelles centrales à fusion seraient beaucoup moins polluantes et sans danger pour l'Homme.
    }
\end{question}
\subquestion{La représentation symbolique du noyau de l'atome X est : $^Z_AX$. Donnez la signification et le nom des nombres Z et A.}
\subquestion{L'hydrogène a trois isotopes : l'hydrogène 1, le deutérium $^1_2H$ et le tritium $^1_3H$. Donnez leur composition.}
\subquestion{Identifiez la réaction qui correspond à une fusion}
\begin{minipage}[c]{0.45\textwidth}
    \begin{center}
        Réaction 1 : \hspace{10pt}
        $^1_3H → ^3_2He + ^0_{-1}e^- $
    \end{center}
\end{minipage}
\hspace{0.1\textwidth}
\begin{minipage}[c]{0.45\textwidth}
    \begin{center}
        Réaction 2 :\hspace{10pt}
        $^2_1H + ^3_1H → ^4_2He + ^1_0n$
    \end{center}

\end{minipage}
% \subquestion{}
%----------------------------------------------------------------------------------------
%	QUESTION 
%----------------------------------------------------------------------------------------
\vspace{10pt}

\begin{minipage}[c]{0.6\textwidth}
    \begin{question}
        \questiontext{
            Une des réactions qui se produit dans les étoiles est la réaction « triple alpha », qui est à l'origine de la formation des noyaux de carbone 12. Cette réaction se produit vers la fin de vie d'une étoile, quand la température (100 MK) devient suffisamment élevée pour que le béryllium 8 puisse rencontrer un noyau d'hélium et former le carbone 12 très stable.
        }
    \end{question}
\end{minipage}
\hspace{0.05\textwidth}
\begin{minipage}[c]{0.35\textwidth}
    \includegraphics[width=0.8\columnwidth]{ex.png}
\end{minipage}

\subquestion{Donnez la composition du noyau d'hélium 4 ($Z_{He}$=2), du béryllium 8 ($Z_{Be}$=4) et du carbone 12 ($Z_C$=6).}
\subquestion{Écrivez les deux équations des réactions qui permettent de transformer l'hélium 4 en carbone 12.}
\subquestion{Indiquez de quel type de réaction il s'agit.}


%----------------------------------------------------------------------------------------
%	QUESTION 
%----------------------------------------------------------------------------------------

\begin{question}
    \questiontext{
        On fait la datation au Carbone 14 d'un morceau de charbon.
        Cet élément radioactif est présent dans tout être vivant à un
        taux constant.
        À leur mort, les échanges de matière avec le milieu n'ayant plus lieu, le taux de carbone 14 diminue car il se désintègre.
        La mesure de ce taux dans un échantillon permet donc de dater
        approximativement sa mort.

        \textbf{Déterminez} l'âge du morceau de charbon sachant que l'activité de l'échantillon testé montre que le nombre d'atomes de carbone 14 est 16 fois plus faible qu'à sa formation.
        (La demi-vie du carbone 14 est de $t\frac{1}{2}$= 5 734 ans.)
    }
\end{question}
%	\answerbox{}



\end{document}