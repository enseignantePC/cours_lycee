%%%%%%%%%%%%%%%%%%%%%%%%%%%%%%%%%%%%%%%%%
% Cleese Assignment (For Students)
% LaTeX Template
% Version 2.0 (27/5/2018)
%
% This template originates from:
% http://www.LaTeXTemplates.com
%
% Author:
% Vel (vel@LaTeXTemplates.com)
%
% License:
% CC BY-NC-SA 3.0 (http://creativecommons.org/licenses/by-nc-sa/3.0/)
% 
%%%%%%%%%%%%%%%%%%%%%%%%%%%%%%%%%%%%%%%%%

%----------------------------------------------------------------------------------------
%	PACKAGES AND OTHER DOCUMENT CONFIGURATIONS
%----------------------------------------------------------------------------------------



\documentclass[24pt]{article}

\input{activite.sty} % Include the file specifying the document structure and custom commands
\input{annotate_equations.sty}


\newcommand{\prgrmparun}{
    \mypage{
        \begin{itemize}
            \item Citer les différentes formes d’énergie utilisées dans les
                  domaines de la vie courante, de la production et des
                  services.
            \item Distinguer les formes d’énergie des différentes sources
                  d’énergie associées.
        \end{itemize}
    }
}


\newcommand{\prgrmpardeux}{
    \mypage{
        \begin{itemize}
            \item Énoncer et exploiter la relation entre puissance, énergie
                  et durée.
            \item Évaluer et citer des ordres de grandeur des puissances
                  mises en jeu dans les secteurs de l’énergie, de l’habitat,
                  des transports, des communications, etc.
        \end{itemize}
    }
}

\newcommand{\prgrmpartrois}{
    \mypage{
        \begin{itemize}
            \item Identifier les principales conversions d’énergie :
                  électromécanique, photoélectrique, électrochimique,
                  thermodynamique (conversions réalisées par une
                  machine thermique), etc.
            \item Schématiser une chaîne énergétique ou une conversion
                  d’énergie en distinguant formes d’énergie, sources
                  d’énergie et convertisseurs.
            \item Évaluer ou mesurer une quantité d’énergie transférée,
                  convertie ou stockée.
        \end{itemize}
    }
}

\newcommand{\prgrmparquatre}{
    \mypage{
        \begin{itemize}
            \item Énoncer le principe de conservation de l’énergie pour un
                  système isolé.
            \item Exploiter le principe de conservation de l’énergie pour
                  réaliser un bilan énergétique et calculer un rendement
                  pour une chaîne énergétique ou un convertisseur.
            \item Déterminer le rendement d’une chaîne énergétique ou
                  d’un convertisseur
        \end{itemize}
    }
}


\newcommand{\prgrmparcinq}{
    \mypage{
        \begin{itemize}
            \item Énoncer qu’une ressource d’énergie est qualifiée de
                  « renouvelable » si son renouvellement naturel est
                  assez rapide à l’échelle de temps d’une vie humaine.
        \end{itemize}
    }
}


\newcommand{\mypage}[1]{
    \begin{minipage}[t]{0.6\textwidth}
        {#1}
    \end{minipage}
}

%	\setlength{\arrayrulewidth}{0.5mm}
%	\setlength{\tabcolsep}{18pt}
\newcommand{\programme}{
    \renewcommand{\arraystretch}{2}
    \begin{center}
        \begin{tabular}{@{}|l|l|@{}}
            \multicolumn{2}{c}{Programme : Énergie}                     \\ \midrule

            \begin{minipage}[t]{0.3\textwidth}
                {Savoirs}\end{minipage}         & \mypage{Savoirs-faire}    \\\midrule

            Formes d’énergie.                          & \prgrmparun    \\

            Énergie et puissance.                      & \prgrmpardeux  \\

            \begin{minipage}[t]{0.3\textwidth}
                Les conversions et les chaînes
                énergétiques.

                Stockage de l’énergie.\end{minipage}       & \prgrmpartrois \\

            \begin{minipage}[t]{0.3\textwidth}
                Principe de la conservation de l’énergie.
                Rendement\end{minipage}  & \prgrmparquatre                  \\

            Ressource d’énergie dite « renouvelable ». & \prgrmparcinq  \\

            \bottomrule\end{tabular}
    \end{center}

}



%----------------------------------------------------------------------------------------
%	VARIABLES
%----------------------------------------------------------------------------------------

% % Required
% \newcommand{\assignmentQuestionName}{Question} % The word to be used as a prefix to question numbers; example alternatives: Problem, Exercise
\newcommand{\assignmentClass}{Physique Chimie ---} % Course/class
\newcommand{\assignmentTitle}{Chapitre 1: Les éléments chimiques} % Assignment title or name
% \newcommand{\assignmentAuthorName}{Mme Cercy} 
% % \newcommand{\assignmentAuthorName}{Chapitre Y} 
% % \newcommand{\titre}{Activité 1 : Les éléments chimiques dans les étoiles} % titre de l'activité
% \newcommand{\titreActivite}{\huge Act 2 : La radioactivité, un phénomène naturel} % titre de l'activité

%----------------------------------------------------------------------------------------

\begin{document}
\thispagestyle{fancy}
\cfoot{}
\date{}
\title{\huge Chapitre 1: Les éléments chimiques}
\maketitle % Print the title page

\programme

\assignmentSection{Cours}


\begin{concept}{Désintégration des noyaux radioactifs }
    \begin{center}
        \textbf{La radioactivité}
    \end{center}
    \begin{list}{$\bullet$}{}
        \item
              \begin{minipage}[c]{0.6\textwidth}
                  Certains noyaux sont instables : on dit qu’ils sont radioactifs.
                  La radioactivité est un phénomène naturel, qui résulte de la transmutation d’un noyau
                  en un autre.

                  Ainsi, les désintégrations successives ont contribué à la
                  formation des 94 éléments chimiques que l’on trouve sur Terre.

                  Un noyau radioactif peut spontanément se désintégrer en émettant soit une particule $α$ (noyau d’hélium),
                  soit un électron, soit un positon.
              \end{minipage}
              \hspace{0.05\textwidth}
              \begin{minipage}[c]{0.25\textwidth}
                  \begin{center}
                      \includegraphics[width=\columnwidth]{nuclear3.png}
                  \end{center}
              \end{minipage}

              \vspace{10pt}
        \item La radioactivité est aléatoire, inéluctable, spontanée et indépendante de
              la substance dans laquelle le noyau radioactif se trouve.
    \end{list}

    \begin{center}
        \textbf{Évolution du nombre de noyaux et demi-vie }
    \end{center}

    \begin{minipage}[c]{0.7\textwidth}
        La population de noyaux d’un échantillon décroît au cours du temps,
        elle est divisée par deux au bout d’une durée appelée « demi-vie ».

        La courbe à côté représente l’évolution du nombre N de noyaux radioactifs en fonction
        du temps t. La désintégration suit une loi mathématique de décroissance.
        $N_0$ représente le nombre de noyaux radioactifs à l’instant $t_0$ (origine des dates) ;
        $t_{\frac{1}{2}}$ représente la demi-vie.
    \end{minipage}
    \hspace{0.05\textwidth}
    \begin{minipage}[c]{0.2\textwidth}
        \begin{center}
            \includegraphics[width=1.2\columnwidth]{nuclear4.png}
        \end{center}
    \end{minipage}
\end{concept}


\begin{concept}{Les éléments chimiques dans l’Univers}
    \begin{minipage}[c]{0.5\textwidth}
        \begin{center}
            \textbf{La composition chimique de l’Univers}
        \end{center}
        \begin{list}{$\bullet$}{}
            \item L’Univers est formé de 118 éléments chimiques différents.
                  L’hydrogène $_1^1$H est l’élément chimique le plus abondant :
                  il représente à lui seul près de 75 \%
                  des atomes présents dans l’Univers.

            \item Sur Terre, on a observé 94 éléments chimiques
                  à l’état naturel, 24 autres ont été créés artificiellement.
        \end{list}
    \end{minipage}
    \hspace{0.05\textwidth}
    \begin{minipage}[c]{0.4\textwidth}
        \begin{center}
            \textbf{La répartition des éléments chimiques}
        \end{center}

        Les éléments sont répartis de manière inégale dans l’Univers :
        on trouve majoritairement de
        l’hydrogène et de l’hélium dans les étoiles, tandis que la Terre
        est formée principalement d’oxygène et de silicium.
    \end{minipage}
\end{concept}




\begin{concept}{Les réactions nucléaires}
    \begin{center}
        \textbf{La fusion, à l’origine de la synthèse des noyaux}
    \end{center}

    \begin{list}{$\bullet$}{}
        \item
              Selon les théories les plus récentes, les premiers atomes ont été formés quelques minutes
              après le « Big Bang ». L’Univers était alors extrêmement chaud ($10^9$ K) et dense,
              les particules élémentaires se sont agglomérées pour former des noyaux d’hydrogène ($^1_1$H), de
              deutérium ($^2_1$H ou $^2_1$D) et d’hélium et de lithium.

              Cette réaction nucléaire est appelée fusion nucléaire.
              \vspace{10pt}
        \item
              \begin{minipage}[c]{0.6\textwidth}
                  Au cours d’une réaction de fusion, des noyaux légers forment un noyau plus lourd en éjectant
                  une particule et en libérant de l’énergie. L’équation de la fusion du deutérium avec le
                  tritium s’écrit : $$^2_1H + ^3_1H \rightarrow ^4_2He + ^1_0n $$

                  La réaction libère une énergie de 3,5 MeV.
              \end{minipage}
              \hspace{0.05\textwidth}
              \begin{minipage}[c]{0.3\textwidth}
                  \begin{center}
                      \includegraphics[width=0.8\columnwidth]{nuclear1.png}
                  \end{center}
              \end{minipage}
    \end{list}

    \begin{center}
        \textbf{Les réactions nucléaires au cœur des étoiles}
    \end{center}

    \begin{list}{$\bullet$}{}
        \item Les autres éléments sont formés au sein des étoiles, formées par accrétion des
              atomes créés lors du Big Bang.
              Les noyaux légers fusionnent et produisent des noyaux plus lourds.
              On y trouve ainsi plusieurs éléments comme l’oxygène (Z=8), le carbone (Z=6),
              mais aussi des noyaux plus lourds comme le fer (Z=26).

        \item Sous l’impact de neutrons ou d’autres particules légères, certains noyaux se
              cassent : c’est la fission.

        \item Ces éléments chimiques sont dispersés à la fin de la vie de l’étoile.
    \end{list}

    \begin{center}
        \textbf{La fission nucléaire}
    \end{center}

    Au cours d’une réaction de fission, des noyaux lourds se cassent en deux noyaux
    plus légers sous l’impact d’un neutron ou d'un proton.
    La réaction s’accompagne de l’éjection d’une particule et libère de l’énergie.
    L’équation de la fission de l’azote bombardé par un proton s’écrit :
    $$^{15}N + ^1H \rightarrow ^{12}C + ^4He$$
    La réaction libère une énergie de 4,96 MeV.

    \begin{center}
        \includegraphics[width=0.4\columnwidth]{nuclear2.png}
    \end{center}

\end{concept}



\end{document}