%%%%%%%%%%%%%%%%%%%%%%%%%%%%%%%%%%%%%%%%%
% Cleese Assignment (For Students)
% LaTeX Template
% Version 2.0 (27/5/2018)
%
% This template originates from:
% http://www.LaTeXTemplates.com
%
% Author:
% Vel (vel@LaTeXTemplates.com)
%
% License:
% CC BY-NC-SA 3.0 (http://creativecommons.org/licenses/by-nc-sa/3.0/)
% 
%%%%%%%%%%%%%%%%%%%%%%%%%%%%%%%%%%%%%%%%%

%----------------------------------------------------------------------------------------
%	PACKAGES AND OTHER DOCUMENT CONFIGURATIONS
%----------------------------------------------------------------------------------------

\documentclass[10pt]{article}
\input{activite.sty} % Include the file specifying the document structure and custom commands
% \usepackage{activite}
%----------------------------------------------------------------------------------------
%	ASSIGNMENT INFORMATION
%----------------------------------------------------------------------------------------

% Required
\newcommand{\assignmentQuestionName}{Question} % The word to be used as a prefix to question numbers; example alternatives: Problem, Exercise
\newcommand{\assignmentClass}{Physique Chimie} % Course/class
\newcommand{\assignmentTitle}{Activité n°3} % Assignment title or name
\newcommand{\assignmentAuthorName}{Mme Cercy} %

%----------------------------------------------------------------------------------------
%	VARIABLES
%----------------------------------------------------------------------------------------

\newcommand{\titreActivite}{Activité 3: La loi de Wien} % titre de l'activité

\begin{document}

\date{}
\title{\titreActivite}
\maketitle % Print the title page
% \begin{center}
\begin{center}
	Le Soleil comme tout corps matériel émet des ondes électromagnétiques. Ces dernières sont interceptées par la surface de la Terre. Leur étude permet de déterminer la température de surface de notre étoile.

	\textbf{$→$ Comment l’étude du spectre d’émission du Soleil permet de déterminer sa température de surface ?}


	\includegraphics[scale=0.45]{doc1/0.png}

\end{center}

\begin{minipage}[t]{0.45\textwidth}
	\centering
	\includegraphics[scale=0.45]{doc1/1.png}
\end{minipage}
\hspace{0.1\textwidth}
\begin{minipage}[t]{0.45\textwidth}
	\centering
	\includegraphics[scale=0.45]{doc1/2.png}
\end{minipage}

\begin{center}
	\includegraphics[scale=0.45]{doc2.png}

\end{center}



\begin{enumerate}
	\item  Doc. 1 Identifiez la longueur d’onde $λ_{max}$ pour laquelle le soleil émet le plus d’énergie.
	\item Doc. 1 À partir de la question précédente, déduisez graphiquement la température de la surface du Soleil. Retrouvez cette valeur par le calcul.
	\item Doc. 2 Déduisez-en le type d’étoile auquel appartient le Soleil selon la classification de Harvard.
\end{enumerate}

\end{document}
