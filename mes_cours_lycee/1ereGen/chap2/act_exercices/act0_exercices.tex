%%%%%%%%%%%%%%%%%%%%%%%%%%%%%%%%%%%%%%%%%
% Cleese Assignment (For Students)
% LaTeX Template
% Version 2.0 (27/5/2018)
%
% This template originates from:
% http://www.LaTeXTemplates.com
%
% Author:
% Vel (vel@LaTeXTemplates.com)
%
% License:
% CC BY-NC-SA 3.0 (http://creativecommons.org/licenses/by-nc-sa/3.0/)
% 
%%%%%%%%%%%%%%%%%%%%%%%%%%%%%%%%%%%%%%%%%

%----------------------------------------------------------------------------------------
%	PACKAGES AND OTHER DOCUMENT CONFIGURATIONS
%----------------------------------------------------------------------------------------

\documentclass[10pt]{article}
\input{activite.sty} % Include the file specifying the document structure and custom commands
% \usepackage{activite}
%----------------------------------------------------------------------------------------
%	ASSIGNMENT INFORMATION
%----------------------------------------------------------------------------------------

% Required
\newcommand{\assignmentQuestionName}{Question} % The word to be used as a prefix to question numbers; example alternatives: Problem, Exercise
\newcommand{\assignmentClass}{Physique Chimie} % Course/class
\newcommand{\assignmentTitle}{Exercice} % Assignment title or name
\newcommand{\assignmentAuthorName}{Mme Cercy} %

%----------------------------------------------------------------------------------------
%	VARIABLES
%----------------------------------------------------------------------------------------

\newcommand{\titreActivite}{\huge Exercices} % titre de l'activité

\begin{document}

\date{}
\title{\titreActivite}
\maketitle % Print the title page
%%%%

\begin{center}
	Exercice 1
\end{center}


\begin{minipage}[c]{0.45\textwidth}
	% \centering
	\includegraphics[scale=0.45]{doc.png}
\end{minipage}
\hspace{0.1\textwidth}
\begin{minipage}[c]{0.45\textwidth}
	% \centering
	\includegraphics[scale=0.45]{graph.png}
\end{minipage}


\begin{enumerate}
	\item  Repérez la longueur d’onde pour laquelle l’intensité lumineuse émise par Sirius est maximale pour chacune des courbes.

	      La courbe noire est celle qui permet de déterminer la température de l'étoile.

	\item  À partir du résultat de la question précédente et de la loi de Wien, concluez sur la température approximative de la surface de Sirius.
\end{enumerate}

%%%%

\begin{center}
	Exercice 2 \vspace{10pt}
\end{center}

\includegraphics[scale=0.4]{ex_stefan_boltzmann/enonce.png}
\includegraphics[scale=0.36]{ex_stefan_boltzmann/volume.png}

\vspace{10pt}

\begin{minipage}[c]{0.45\textwidth}
	\centering
	\includegraphics[scale=0.45]{ex_stefan_boltzmann/graph.png}
\end{minipage}
\hspace{0.1\textwidth}
\begin{minipage}[c]{0.45\textwidth}
	\centering Données concernant le Soleil.
	\includegraphics[scale=0.45]{ex_stefan_boltzmann/carte.png}

	\begin{enumerate}
		\item En considérant le Soleil comme un corps noir, calculez sa température de surface.
		\item À l'aide de la loi de Stefan-Boltzmann, calculez la puissance surfacique du rayonnement solaire.
		\item Déduisez des calculs précédents et de l'énoncé la puissance du rayonnement solaire. Comparez la valeur obtenue avec celle donnée dans l'activité 1.
	\end{enumerate}
\end{minipage}

\clearpage


\begin{center}
	Exercice 3
\end{center}


\begin{minipage}[c]{0.45\textwidth}
	% \centering
	Une éclipse totale de Soleil est un événement rare et impressionnant.

	L'éclipse du 11 août 1999 est considérée comme celle ayant été suivie par le plus d'êtres humains dans toute l'histoire :
	en effet, elle a traversé des zones très fortement peuplées comme l'Europe et l'Asie.
	Une large partie de l'hémisphère nord a été plongée, à un moment de la journée, dans la pénombre. En revanche, seule une
	petite partie de la surface terrestre a été plongée dans une totale obscurité : une tache complètement noire se déplaçait
	sur Terre, donnant ainsi aux quelques chanceux sur place la sensation étrange d'observer la tombée de la nuit (puis le lever du jour)
	en quelques secondes à peine. C'est cette « tache d'ombre » qui va nous intéresser tout au long de cet exercice.

\end{minipage}
\hspace{0.05\textwidth}
\begin{minipage}[c]{0.45\textwidth}
	\centering
	\includegraphics[scale=0.3]{eclipse/photo.png}
\end{minipage}
\vspace{5pt}

\begin{enumerate}
	\item Le centre de la tache d'ombre est notamment passé au nord-est de Paris, à Reims.
	      Cette tache de 112 km de diamètre se déplaçait d'ouest en est à une vitesse de 3 055 km·h-1.
	      Déterminez la durée pendant laquelle les Rémois (habitants de Reims) se sont retrouvés plongés dans l'obscurité.

	\item La puissance émise par le Soleil est interceptée par la Terre.
	      Rappelez sur quelle surface cette puissance est interceptée, puis déterminez l'expression
	      littérale et la valeur numérique de son aire.

	\item Déterminez la valeur numérique de l'aire de la tache d'ombre.

	\item Normalement, la Terre intercepte une puissance émise par le Soleil égale à 1,74 × $10^{17}$ W.

	      À l'aide des questions précédentes, déterminez la puissance $P_{ombre}$ que la Terre ne peut pas intercepter
	      à cause de la présence de la tache d'ombre.

	\item Déterminez la valeur numérique de l'énergie que la Terre n'a pas pu intercepter à cause de la tache
	      d'ombre pendant toute la durée où Reims s'est trouvée plongée dans l'obscurité.

	\item La consommation électrique annuelle rémoise s'élève à environ 4 × $10^5$ MWh.
	      Combien de temps faudrait-il aux Rémois pour consommer autant d'énergie que la valeur calculée à la question précédente ?
\end{enumerate}


\begin{center}
	Exercice 4 \vspace{10pt}

	\includegraphics[scale=0.5]{ex_lune.png}
\end{center}



\end{document}



